% Document class `report-template` accepts either project-plan or final-report option in
% []. This will change the title page as necessary.
\documentclass[project-plan]{report-template}
%\documentclass[final-report]{report-template}

% Packages I use in my report.
\usepackage{graphicx}
\usepackage{amsmath}
\usepackage{blindtext}
\usepackage{array}

% Directory where I saved my figures.
\graphicspath{{./figures/}}

% Metadata used for the title page
\university{Imperial College London}
\department{Department of Earth Science and Engineering}
\course{MSc in Environmental Data Science and Machine Learning}
\title{Noise reduction in seismic images, investigating deep learning from natural image models}
\author{Robert Smith}
\email{robert-edward-duncan.smith23@imperial.ac.uk}
\githubusername{edsml-rs4623}
\supervisors{James Lowell, Geoteric\\
             Prof. Rebecca Bell}
\repository{https://github.com/ese-msc-2023/irp-rs4623}

\begin{document}

\maketitlepage  % generate title page

% Abstract
\section*{Abstract}
Seismic image denoising is an key exploration area, both for research and commercial purposes, as it is crucial for improving the understand of subsurface geological features which are highly significant for such purposes as Carbon capture, Utilisation and Storage. This project looks to extend current commercial application by exploring deep learning approaches to denoising, utilising Geoteric's in-house synthetic data model. Specifically the viability of leveraging existing insight and architectures from the significant amount of research in natural image denoising. It looks to build on existing applications of deep learning to seismic image denoising, and evaluating and develop training schemes. Ultimately looking to progress the development of automated denoising products.

\newpage
\tableofcontents
\newpage

% Introduction section
\section{Introduction}
High quality seismic images are crucial for understanding underground geological structures. They facilitate qualitative and quantitative interpretation and enable further analysis such as automated feature recognition.

Owing to the complex nature of the Earth's subsurface and the seismic image construction process noise inevitably contaminates the data and can significantly affect its quality. This noise can be caused by many factors during both the reflection data acquisition process, for example from environmental noise, as well as during processing, such as migration. Various factors contribute to seismic noise, but they can be classified into two broad types: coherent noise and random noise. 

There are a multitude of traditional methods for attenuating noise that have been used. Some of these methods include; filter-based strategies, such as mean filtering~\cite{Liu20062d} or f-x de-convolution ~\cite{hornbostel1991spatial}; spare transformation-based strategies, such as using Fourier transforms ~\cite{alsdorf1997noise} or wavelet transform ~\cite{Albert2009wavelet}; and modal decomposition techniques, such as empirical mode decomposition ~\cite{bekara2009random} and singular value decomposition ~\cite{bekara2007local}. These methods have differing drawbacks such as they are not adaptive, applying consistently across the data, and they often rely on expert experience for manual parameter selection.

The drawbacks of traditional methods, combined with advances in deep learning frameworks and available computational power, have led to the application of deep learning models for seismic denoising.

Whilst various application of deep learning frameworks to seismic data have been explored, there has been little in the way of utilising the more current architectures, and their trained models, proven effective on natural images. Given the ability for modern deep neural networks to effectively extract multi-scale features for natural images and produce models that can generalise across domains. The challenge then is to to explore the progress in deep learning frameworks for natural datasets, and their applicability to seismic data. Investigating and evaluating the effectiveness of different transfer learning approaches and training strategies and utilise Geoteric's in-house synthetic data model to produce training data.

Improving the noise removal process is a significant area for research as it enables more accurate imaging and thus improved interpretation of the geological structures. This work seeks to further develop the area of seismic image de-noising using deep learning, and contribute towards the commercial development of denoising/restoration products. High quality seismic data also facilitate down-stream applications such as for identifying appropriate sites Carbon Capture, Utilisation and Storage.

\subsection{Literature review}
\subsubsection{Model architectures}
The use of neural networks, specifically convolutional neural networks (CNNs), have led the advancements in learning models for Image restoration and denoising over recent years. The use of residual blocks in CNNs ~\cite{he2015DeepRL} allowed gradients to flow through the block helping to address the issue of vanishing gradients facilitating the use of much deeper networks. In image restoration tasks where the original mapping is closer to an identity mapping, the residual mapping will be much easier to optimise and can improve performance. Residual blocks were then implemented in the influential denoising convolutional neural network (DcCNN) ~\cite{zhang2017beyond}, which lead to significantly increased denoising performance than traditional Gaussian denoisers. 

Some CNN models were then developed to be fast and flexible in their application, ~\cite{zhang2018ffdnet}. However, these non-blind denoisers require knowledge of the image noise, which can reduce performance for real images where spatially variant noise is hard to estimate.  

In order to create models that generalise well, blind models incorporated a noise model, and contained a sub-network to learn the noise and another non-blind denoiser ~\cite{Guo2019Cbdnet}. The model generalised better and when trained on synthetic and real noisy images they improved performance against benchmarks. However these models are dependent on developing a good noise model and having real noisy data with ground-truths, both of which can be challenging.

In order to allow the model to selectively focus on the important parts of the input data various attention mechanisms have been implemented into computer vision models ~\cite{wang2017residual}~\cite{hu2018senet}~\cite{woo2018cbam}~\cite{wang2020eca}~\cite{hou2021coordinate}. Incorporating simple feature attention has improved the performance of Denoising networks ~\cite{anwar2019rid}. Combining attention mechanisms, for example to enhance the spatial and channel representation of the model, have be shown to improve networks ~\cite{cai2023cbamdncnn}.

Recently more complex architectures that combine attention with capturing multi-scale resolution, aiming to capture both high resolution spatially precise representations along with maintaining low resolution representations with contextual information, have achieved state of the art results show in ~\ref{fig:example} ~\cite{zamir2020MIRNet}. Results from Multi-stage models based on the ides of progressively learning to capture contextual and spatial features~\cite{zamir2021MPRNet} have also achieved compelling results, however these networks are very complex and are more computationally expensive.

% Figure
\begin{figure}[htb]
    \begin{center}
        \includegraphics[width=0.4\textwidth]{latex/figures/noise_removal_ex.png}
    \end{center}
    \caption{\label{fig:example} Example denoising results from CNN network.}
\end{figure}

Most DL applications for seismic denoising apply Dc-CNN based architectures, based on simple U-Net structures ~\cite{li2021seismicdccnn}~\cite{li2021seismicdlsims}. The use of more complex state of the art CNNs, for example incorporating attention, has not been widely explored.

Transformer based image restoration models, leveraging self attention mechanisms to capture long range pixel interactions effectively, have also produced state of the art results for denosing~\cite{zamir2021restormer}~\cite{wang2022uformer}. Very recently, there has been some use for seismic noise attenuation ~\cite{lei2024swin}. Transformer networks computational requirements can increase drastically with increased in image resolution, a potential fix has been to incorporate  self-attention within local regions using Swin transformer design ~\cite{wang2022uformer}. Further designs have improved the context aggregation capability of these networks~\cite{zamir2021restormer}.


\subsubsection{Training schemes}
Supervised deep learning models, like the above, can perform well on denoising then but require a realistic noisy and clean image pairs to learn mappings, and a significant amount of realistic data to be able to generalise. Self supervised learning strategies can then help resolve some of these challenges as they don't require clean images for training.
Noise2Noise~\cite{lehtinen2018noise2noise} pioneered this approach by denoising images using solely noisy images to train and producing performance similar to existing using clean labels. Many other self-supervised strategies have been developed including, Noise2void~\cite{krull2019noise2void}, Self2self~\cite{quan2020self2self}.

Given the challenge of having real noisy and clean data for seismic models, thus reliance on synthetic data, self-supervised models have recently advanced the field of seismic noise attenuation~\cite{liu2023tracewise}. These networks often rely on a random noise assumption~\cite{birnie2021potential}, or are capable in suppressing coherent noise in  a specific direction~\cite{liu2023tracewise}, but they don't perform as well with real dynamic noise properties.

Deep Image Priors (DIP), are another unsupervised method which has been applied to seismic image ~\cite{lie2024seismicdip} with good results. The networks themselves act as the implicit image prior, the structure inherently favours natural structures over random noise. Whilst they don't require pre-training they have to be trained for each image and require some optimisation, for example early stopping to stop over-fitting, which complicated their practical use.

There has been some investigating into different learning strategies but little into the use of feature learning from pre-trained models that produce high fidelity with natural images.


\subsection{Objectives}
The project aims to investigate using established and proven model architectures alongside Geoteric's in-house synthetic seismic generation model. Specifically, learning from natural image denoising by leveraging models and techniques that have proved successful. The ultimate objective is to developing a deep learning seismic denoising model that effectively removes noise, and improves the performance of their existing methods. Potentially expanding to 3D seismic sections if model performance justifies and time allows.


\section{Methodology}
As per the project brief there has been a significant amount of research and advances in the application of Deep learning to natural image datasets, including on models to denoise and improve image quality.

I intend to use a state of the art deep CNN with residual attention mechanism, such a MIRNetV2~\cite{zamir2022mirnetv2}, using synthetic data in a structured training scheme. I intend to conduct ablation studies for different loss functions and training schemes, in order to understand their impact. Using SNR and SSIM as key evalutation metrics.

One limitation is that there are other models such as transformers which have also state of the art performance for denoising, however, these models can be computationally much more expensive and currently do not lead to significantly better results.  Another limitation is the use of structured training with synthetic data. Synthetic data can lack many of the realistic features embedded in real data, and contain noise which effectively mimics that of field data. Although it has shown promising capabilities on synthetic datasets, performance can drop with generalisation to field data ~\cite{zhang2019unsupervised}. However synthetic data from Geoteric's proprietary model represents real data much more accurately, and has proved successful in other DL related tasks.

\section{Expected Outcomes}
The project looks to demonstrate the viability of using established deep learning architectures, along with proprietary synthetic data generation, to produce models that can used for denoising products. Given the breadth and depth of the current research the models may not produce state of the art results, however this is not necessary for a sucessful outcome.

\section{Future plan}
\begin{table}[h]
\centering
\begin{tabular}{|c|c|c|}
\hline
\textbf{Milestone} & \textbf{Timeline} & \textbf{Status} \\
\hline
Review natural image and seismic denoising literature & 14th June 2024 & Completed \\
\hline
Familiarise with implementation of existing and pre-trained models & 21st June 2024 &  In progress \\ 
\hline
Develop methodology to evaluate performance & 30th June 2024 & Not started \\
\hline
Evaluate performance on 2D images  & 31st July 2024 & Not started \\
\hline
Update implementation and evaluate for 3D images  & 12th August 2024 & Not started \\
\hline
Develop and train final models & 16th August 2024 & Not started \\
\hline
\end{tabular}
\end{table}

% References
\bibliographystyle{plain}
\bibliography{references}  % BibTeX references are saved in references.bib

\end{document}          
